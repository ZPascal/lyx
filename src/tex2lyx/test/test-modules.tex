%% LyX 1.6.1 created this file.  For more info, see http://www.lyx.org/.
%% Do not edit unless you really know what you are doing.
\documentclass[oneside,english]{amsart}
\usepackage[T1]{fontenc}
\usepackage[latin9]{inputenc}
\usepackage{amsthm}

%%%%%%%%%%%%%%%%%%%%%%%%%%%%%% Textclass specific LaTeX commands.
\numberwithin{equation}{section} %% Comment out for sequentially-numbered
\numberwithin{figure}{section} %% Comment out for sequentially-numbered
\providecommand{\theoremname}{Theorem}
\theoremstyle{plain}
\theoremstyle{plain}
\newtheorem{thm}{\protect\theoremname}
  \theoremstyle{plain}
  \newtheorem{lem}[thm]{Lemma}

\usepackage{babel}

\begin{document}
This is a dummy file

It has a theorem, a lemma and a proof.

The theorem is recognized is a style provided by the module theorems-ams,
since the preamble code matches.

The lemma is not recognized as a command provided by a module, since the
preamble code is from an older version of LyX, and modules are only loaded
if the preamble code matches (otherwise you could easily get completely
different output for some often used names like \textbackslash theorem.

The proof is recognized as a builtin style provided by the text class.

\begin{lem}
this is a lemma\end{lem}
\begin{thm}
this is the theorem\end{thm}
\begin{proof}
this is the proof
\end{proof}

\end{document}
